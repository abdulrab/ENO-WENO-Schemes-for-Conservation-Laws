\documentclass{beamer}
\usepackage[siunitx,europeanresistors,americaninductors]{circuitikz}% elektriskās shēmas
\usepackage{tikz}% priekš elektriskajām shēmām kaut kas
\usepackage{amsmath}% var izmantot math mode
\usepackage{color}% var mainīt krāsu teksta paragrāfa vidū
\usepackage{wrapfig}% attelam apkart iet teksts
\usepackage{blindtext}% labākais veids kā saprast erorrus
\usepackage[utf8]{inputenc}
\usepackage{graphicx} % var ielikt bildes
\usepackage{tabularx} % tabulas
\usepackage{multicol} % vairākas kolonas var izmantot
\usetheme{Goettingen} % Uzliek labjā malā nosaukumu un autoru
% \usetheme{default} % Uzliek labjā malā nosaukumu un autoru

\title{High Order Essentially Non-Oscillatory Schemes and Weighted Essentially Non-Oscillatory Schemes for Selected Riemann Problems}
% \author{Abdul Rab}
\author[Abdul Rab]{Abdul Rab\\[10mm]}
\institute{Sukkur IBA University}
\date{07 February 2022}
 
\begin{document} 
\frame{\titlepage} 

\section{Introduction}

\begin{frame}{Definitions}
    \begin{enumerate}
        \item Hyperbolic PDEs
        \item Conservation Laws
        \item Consistency
        \item Stability
        \item Convergence
        \item Riemann Problem
        \item Scheme
    \end{enumerate}
\end{frame}


\section{ENO Scheme for Conservation laws}
\begin{frame}{ENO Schemes}
\begin{enumerate}
    \item Finite Volume Method
    \item ENO Scheme
    \item Numerical Fluxes
\end{enumerate}
\end{frame}

\section{Mathematical Model}
\begin{frame}{Mathematical Model}
    Applying the ENO weno Schemes on the Proposed model in the article
    Show the Model with system of equations
\end{frame}


\section{Future Work}
\begin{frame}{Extension}
    Higher Order Multidimensional ENO/Weno scheme for proposed model
    Implementing Algorithm for proposed Mathematical Model.
\end{frame}


%% References
\begin{frame}[allowframebreaks]
        \frametitle{References}
        \bibliographystyle{amsalpha}
        \bibliography{references.bib}
\end{frame}

\end{document}