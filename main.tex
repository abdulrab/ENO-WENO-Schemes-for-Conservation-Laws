\documentclass{beamer}
\usepackage[siunitx,europeanresistors,americaninductors]{circuitikz}% elektriskās shēmas
\usepackage{tikz}% priekš elektriskajām shēmām kaut kas
\usepackage{amsmath}% var izmantot math mode
\usepackage{color}% var mainīt krāsu teksta paragrāfa vidū
\usepackage{wrapfig}% attelam apkart iet teksts
\usepackage{blindtext}% labākais veids kā saprast erorrus
\usepackage[utf8]{inputenc}
\usepackage{graphicx} % var ielikt bildes
\usepackage{tabularx} % tabulas
\usepackage{multicol} % vairākas kolonas var izmantot
% \usetheme{Goettingen} % Uzliek labjā malā nosaukumu un autoru
\usetheme{default} % Uzliek labjā malā nosaukumu un autoru

\title{High Order Essentially Non-Oscillatory Schemes and Weighted Essentially Non-Oscillatory Schemes for Selected Riemann Problems}
% \author{Abdul Rab}
\author[Abdul Rab]{Abdul Rab\\[10mm]}
\institute{Sukkur IBA University}
\date{\today}
 
\begin{document} 
\frame{\titlepage} 

\section{Introduction}

\begin{frame}{Definitions}
    \begin{enumerate}
        \item Hyperbolic PDEs
        
        \item Riemann Problem
        \item Scheme
    \end{enumerate}
\end{frame}
\begin{frame}{Hyperbolic conservation law}
A conservation law states that a particular measurable property of an isolated physical system does not change as the system evolves over time. Hyperbolic PDEs of the following form happens to be the conservation law.
\begin{equation}
    \frac{\partial }{\partial t}u(x,t) + \frac{\partial }{\partial x}f(u(x,t)) = 0
\end{equation}
Here $u(x,t)$ represents the quantity that moves according to the given flux function $f(u(x,t))$.\\

\end{frame}

\section{ENO Scheme}
\begin{frame}{ENO Reconstruction}
The finite volume scheme 

\begin{equation}
    \frac{d}{dt} \bar{u_i} + \frac{1}{\Delta x_i} \left[ \hat{f}(u^{-}_{i+\frac{1}{2}},u^{+}_{{i+\frac{1}{2}}}) - \hat{f}(u^-_{i-\frac{1}{2}} , u^+_{i-\frac{1}{2}} ) \right] = 0
\end{equation}

Consider three three-point stencils
\begin{align}
    \{ I_{j-2},I_{j-1},I_{j} \}, \; \; \{ I_{j-2},I_{j-1},I_{j} \} \; \;  \{ I_{j-2},I_{j-1},I_{j} \}
\end{align}

Stencils with smallest variation (measured by divided difference) is chosen to reconstruction $u^-_{j+\frac{1}{2}}$.


\begin{enumerate}
    \item Finite Volume Method
    \item ENO Scheme
    \item Numerical Fluxes
\end{enumerate}
\end{frame}


\begin{frame}{ENO stencil selection}
    \begin{itemize}
        \item Solution cell averages:
        \begin{equation}
            u[x_j] = \bar{u_j}, \; \; \; \forall j
        \end{equation}
        
        \item First order divided difference:
        \begin{equation}
            u[x_j,x_{j+1}] = \frac{u[x_{j+1}] - u[x_j] }{x_{j+1} - x_j}, \; \; \forall j
        \end{equation}
        \item kth order divided difference:
        \begin{equation}
            u[x_j,x_{j+1},\cdots,x_{j+k}] = 
        \end{equation}
    \end{itemize}
\end{frame}

\begin{frame}{ENO Stencil}
    Figure here
\end{frame}

\begin{frame}{ENO Reconstruction of $u^-_{j+\frac{1}{2}}$}
\begin{enumerate}
    \item Choose ENO stencil containing 3 cells
    \begin{enumerate}
        \item starting with $\bar{u}_j$
        \item choose 3 cells in the ENO stencil, in an adaptive manner guided by divided differences.
    \end{enumerate}
    
    \item Contruct polynomials of degree 2, whose cell averages agree with the given cell averages.
    \begin{equation}
        \int P_2(x)dx = \bar{u_j} \; \; I \in \text{ ENO stencil.}
    \end{equation}
    
    \item Evaluate $P_2(x)$ at $x_{j+\frac{1}{2}}$ approximating $u^-_{j+\frac{1}{2}}$.
\end{enumerate}
\end{frame}

\begin{frame}{ENO Scheme}
    Given $\bar{u_j} \forall j$ 
    \begin{enumerate}
        \item ENO reconstruction $u^{\pm}_{j+\frac{1}{2}},\;\; \forall j$.
        
        \item Evaluate numerical fluxes
        \begin{equation}
            \hat{f}_{j+\frac{1}{2}} = \bar{f}(u^-_{j+\frac{1}{2}}, u^+_{j+\frac{1}{2}}), \;\;\; \forall j.
        \end{equation}
        
        \item Evolve the ODE system by method of lines
        \begin{equation}
            \frac{d}{dt} \bar{u_j} + \frac{1}{\Delta x} \left( \bar{f}_{j+\frac{1}{2}} - \hat{f}_{j+\frac{1}{2}}  \right) = 0,
        \end{equation}
        to update $\bar{u}$ for each RungeKutta stage, and finally $u^{n+1}_j$.
    \end{enumerate}
\end{frame}

\section{Mathematical Model}
\begin{frame}{Mathematical Model}
    Applying the ENO weno Schemes on the Proposed model in the article
    Show the Model with system of equations
\end{frame}


\section{Future Work}
\begin{frame}{Future Work}
    Higher Order Multidimensional ENO/Weno scheme for proposed model
    Implementing Algorithm for proposed Mathematical Model.
\end{frame}


%% References
\begin{frame}[allowframebreaks]
        \frametitle{References}
        \bibliographystyle{amsalpha}
        \bibliography{references.bib}
\end{frame}

\end{document}